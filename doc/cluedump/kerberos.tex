\subsection{Kerberos}

\begin{frame}
  \frametitle{Group locker support}

  \begin{itemize}
  \item ``Users'' on scripts are actually lockers.
  \item User IDs are actually locker volume IDs.
  \item Kerberos is modified to let users SSH in as any locker they
    administrate.
    \begin{itemize}
    \item Replaced the \texttt{.k5login} mechanism:
      \texttt{krb5\_kuserok()} in
      \texttt{krb5/src/lib/krb5/os/kuserok.c}
    \item Calls a Perl script \texttt{/usr/local/sbin/admof} to do the
      actual check.
    \end{itemize}
  \end{itemize}
\end{frame}

\begin{frame}[fragile]
\begin{footnotesize}
\begin{semiverbatim}
 krb5_boolean KRB5_CALLCONV
 krb5_kuserok(krb5_context context, krb5_principal principal,
              const char *luser)
 \{
     \ldots
+    if ((pid = fork()) == -1) \{
+       free(princname);
+       return(FALSE);
+    \}
+    if (pid == 0) \{
+#define ADMOF_PATH "/usr/local/sbin/ssh-admof"
+        exec(ADMOF_PATH, ADMOF_PATH, (char *) luser, princname, NULL);
+        exit(1);
+    \}
+    if (waitpid(pid, &status, 0) > 0 && WIFEXITED(status) &&
+        WEXITSTATUS(status) == 33) \{
+        isok = TRUE;
+    \}
     \ldots
 \}
\end{semiverbatim}
\end{footnotesize}
\end{frame}
